% !TeX program = xelatex
\documentclass{ctexart}

\begin{document}

\section{崩坏星穹铁道角色管理系统——需求分析}

\subsection{一、系统目标}
本系统面向"崩坏星穹铁道"游戏的玩家和策划,提供角色、BOSS、环境等核心数据的高效管理、查询与智能推荐服务。系统需支持多维度角色信息管理、个性化成长记录、标签化推荐与全服统计分析,满足策划与玩家的不同业务需求。

\subsection{二、用户与权限}
用户分为两类:玩家、策划。

策划拥有全局管理权限,可维护所有角色、BOSS、环境及标签数据,并进行数据分析与推荐。

玩家仅能管理和查询自己拥有的角色,并可查询全服角色、BOSS、环境等信息,参与排行榜和推荐。

\subsection{三、核心数据对象及属性}
\begin{itemize}
  \item \textbf{用户(User)}:记录系统登录者的基本信息,包括用户ID、用户名、密码、用户类型(玩家/策划)。
  \item \textbf{阵营(Camp)、属性(Attribute)、命途(Fate)}:作为基础数据表,分别存储所有可选的阵营、属性、命途类型,便于角色和BOSS的规范化管理和扩展。
  \item \textbf{角色(Character)}:记录游戏中所有角色的固有属性,包括名称、简介、阵营、属性、命途、技能等。角色的成长性和个性化数据(如星魂、等级、战力、好感度、获得时间等)由玩家-角色表(User\_Character)单独管理,实现共性与个性数据分离。
  \item \textbf{玩家-角色(User\_Character)}:记录每个玩家拥有的角色及其个性化成长数据,包括星魂、等级、战力、好感度、获得时间等。
  \item \textbf{BOSS(Boss)}:记录所有BOSS的基本信息,包括名称、简介、阵营、弱点(弱点属性通过外键关联属性表)等。
  \item \textbf{环境(Environment)}:记录战斗环境的相关信息,包括环境名称、buff等。
  \item \textbf{标签(Tag)}:用于对角色、BOSS、环境进行多维度分类和推荐,标签类型可区分战斗类型、打法推荐等。
  \item \textbf{多对多关系表}:
    \begin{itemize}
      \item 角色-标签(Character\_Tag):实现角色与标签的多对多关联。
      \item BOSS-标签(Boss\_Tag):实现BOSS与标签的多对多关联。
      \item 环境-标签(Environment\_Tag):实现环境与标签的多对多关联。
    \end{itemize}
\end{itemize}

\subsection{四、主要功能需求}
\begin{enumerate}
  \item 策划端
    \begin{itemize}
      \item 维护和查询全服的角色、BOSS、环境、标签等数据。
      \item 根据已有信息推理和推荐最适合的新角色信息,辅助游戏内容设计与平衡。
    \end{itemize}
  \item 玩家端
    \begin{itemize}
      \item 维护和查询自己的角色库,自动计算角色战力。
      \item 查询全服角色库(区分已获得和未获得)、BOSS库、环境库。
      \item 查询全服角色战力排行、角色持有率等统计信息。
      \item 查询当前BOSS和环境推荐的角色(基于角色战力、标签、好感度等)。
      \item 根据BOSS、环境的标签进行智能推荐。
      \item 推荐玩家抽取什么类型的新角色,辅助玩家决策。
    \end{itemize}
\end{enumerate}

\subsection{五、数据关系与推荐逻辑}
角色、BOSS、环境均可拥有多个标签,标签作为推荐和分类的核心依据。

推荐逻辑通过标签的交集实现:如角色与BOSS/环境拥有相同标签,则该角色被推荐用于该BOSS/环境。

阵营、属性、命途等基础属性通过外键规范化,便于后续扩展和维护。

BOSS的弱点通过外键与属性表关联,保证属性体系统一。

\subsection{六、扩展性与安全性}
支持角色、BOSS、环境属性的进一步细化和扩展。

玩家只能操作和查看自己的角色数据,策划可全局管理,保证数据安全和权限隔离。

系统需支持高效的多条件查询、统计分析和智能推荐。

\subsection{七、总结}
本系统以角色、BOSS、环境、标签为核心,结合多维度属性和灵活的标签机制,实现了数据的规范化管理和智能推荐,满足策划和玩家的多样化需求,并具备良好的扩展性和安全性。

\section{关系模式}
\begin{itemize}
  \item \textbf{用户(User)}
    \begin{description}
      \item[user\_id] 用户ID,主码
      \item[username] 用户名
      \item[password] 登录密码
      \item[user\_type] 用户类型(玩家/策划)
    \end{description}
  \item \textbf{阵营(Camp)}
    \begin{description}
      \item[camp\_id] 阵营ID,主码
      \item[camp\_name] 阵营名称
    \end{description}
  \item \textbf{属性(Attribute)}
    \begin{description}
      \item[attribute\_id] 属性ID,主码
      \item[attribute\_name] 属性名称
    \end{description}
  \item \textbf{命途(Fate)}
    \begin{description}
      \item[fate\_id] 命途ID,主码
      \item[fate\_name] 命途名称
    \end{description}
  \item \textbf{角色(Character,固有属性)}
    \begin{description}
      \item[character\_id] 角色ID,主码
      \item[name] 角色名称
      \item[description] 角色简介
      \item[camp\_id] 阵营ID,外码,参照Camp(camp\_id)
      \item[attribute\_id] 属性ID,外码,参照Attribute(attribute\_id)
      \item[fate\_id] 命途ID,外码,参照Fate(fate\_id)
      \item[skill] 技能
    \end{description}
  \item \textbf{玩家-角色(User\_Character,个性属性)}
    \begin{description}
      \item[user\_id] 用户ID,主码,外码,参照User(user\_id)
      \item[character\_id] 角色ID,主码,外码,参照Character(character\_id)
      \item[star\_soul] 星魂(个性属性)
      \item[level] 等级
      \item[power] 战力
      \item[join\_time] 获得/登场时间
      \item[favor] 好感度
    \end{description}
  \item \textbf{BOSS}
    \begin{description}
      \item[boss\_id] BOSS ID,主码
      \item[name] BOSS名称
      \item[description] BOSS简介
      \item[camp\_id] 阵营ID,外码,参照Camp(camp\_id)
      \item[weakness\_id] 弱点属性ID,外码,参照Attribute(attribute\_id)
    \end{description}
  \item \textbf{环境(Environment)}
    \begin{description}
      \item[env\_id] 环境ID,主码
      \item[name] 环境名称
      \item[buff] 环境buff
    \end{description}
  \item \textbf{标签(Tag)}
    \begin{description}
      \item[tag\_id] 标签ID,主码
      \item[name] 标签名称
      \item[tag\_type] 标签类型(如战斗类型、打法推荐等)
    \end{description}
  \item \textbf{角色-标签(Character\_Tag)}
    \begin{description}
      \item[character\_id] 角色ID,主码,外码,参照Character(character\_id)
      \item[tag\_id] 标签ID,主码,外码,参照Tag(tag\_id)
    \end{description}
  \item \textbf{BOSS-标签(Boss\_Tag)}
    \begin{description}
      \item[boss\_id] BOSS ID,主码,外码,参照Boss(boss\_id)
      \item[tag\_id] 标签ID,主码,外码,参照Tag(tag\_id)
    \end{description}
  \item \textbf{环境-标签(Environment\_Tag)}
    \begin{description}
      \item[env\_id] 环境ID,主码,外码,参照Environment(env\_id)
      \item[tag\_id] 标签ID,主码,外码,参照Tag(tag\_id)
    \end{description}
\end{itemize}

\end{document}
